\documentclass[../main.tex]{subfiles}
\graphicspath{{\subfix{../figures/}}}

\begin{document}

\chapter{{Differential Fractionation}}
Diff Frac: Normalisation of RNA-Seq data to enable comparison across experiments remains a non-trivial task. 

DESeq2, TPM, spike-ins, and normalising genes all have assumptions that can often be broken.

Modelling errors using Bayesian statistical models and implementing quality experimental design can overcome many issues.

Research questions are getting more specific and naucened whilst data sets are dramatically increaing in sized. 

To be able to compare samples across RNA-seq runs users need to be able to remove biases that changes from run to run. For example, the amplificaition of reads from each lane of a RNA-seq machine varies significantly. Therefore direct comparison of transcript counts mapped to a genome will, for the most part, depend on the total reads read by that lane of the RNA-seq machine.  Other amplification biases can be introduced, the shotgun method of short read sequencing means the number of reads per gene will be proportional to the length of that gene. Simply because longer genes can be separated into more fragments. In addition elongation biases between the nucleotides can change the amplificiation ratio of genes. Certein gene with high GC content may not be amplified as well because they bind too tightly for the melting stage to occur efficiently. 

Raw reads are rarely used as the measure of expression between samples and between genes. Instead, they are normalised either by internal methods or by external controls such as spike-in samples. Internal normalisation commonly consists of converting raw reads into transcripts per million or reads per million. It can been widely reported at the read per million method has significant biases and should not be used. However, transcripts per million account for the total read variation between runs and the gene length biases. However, dividing by total reads does leave TPM counts vunerable to being dominated by the behaviour of a few highly expressed genes. Since there is a large order of magnitude between expression across an organisms genome, gene that are expressed on the order of $10^4$ transcripts per cell will contribute more to the total reads compared to transcript that only occur once or twice. If the experiments significant change the expression of the highly expressing genes but the users is mostly interested in the behavour of the lowly expressing genes, comparing TPM may confound expression patterns. Neither method attempts to deal with GC biases. Alternatively, an external control of known volumes of synthetic RNA can be introduced with differing GC content and lengths. The comparison of these RNA levels after being amplified in the RNA-seq assay can discover biases in transcript length, total read and GC content. A rigorous method for using this information to normal counts across the genome has not be developed as error in spike-in volumes are very common and obscure predictions expected reads post-sequencing.

\end{document}