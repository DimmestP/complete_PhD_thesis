\documentclass[../main.tex]{subfiles}
\graphicspath{{\subfix{../figures/}}}

\begin{document}
\chapter{Lay Summary}
The broad understanding of how cells regulate themselves is that DNA is the eternal information storage vessel inherited over generations and encodes the proteins that are the vehicles through which a cell controls itself and responds to its environment. RNA, meanwhile, is considered an auxillary molecule used to facilitate the flow of information from DNA to proteins through the formation of protein creating ribosomes, collection of protein forming amino acids and holding the transcription of DNA sequence that encode the protein. However, RNA predates this model of cells, known as central dogma of molecular biology. It is thought that the ancestors of all known life depended entirely on RNA as both the storage vessel and vehicle of action. It is no surprise then that mechanisms that regulate a cell response through the control of its RNA are regularly found to be shared across distinct branches of the evolutionary tree. 

As the technology to explore the world of RNA has developed over the last 30 years, we have continued to discover the variety and importance of RNA. First off, 90\% of the human genome does not encode any proteins but the list of non-coding RNA transcripts continue to grow. Furthermore, fragments of RNA that were once considered discarded remanence of the messy process to create proteins from DNA sequence are beginning to be understood to have vital regulatory roles in and of themselves. Several distinct populations of RNA are now know but focusing back to those that do encode proteins, messenger RNA (mRNA), many regulatory process have been discovered to control them post-transcription. Ultimately, since the same mRNA transcript can be used to create hundreds of copies of a protein it is far more efficient to control the one mRNA molecule than hundreds of its proteins. Localising mRNA to specific sub-cellular locations ensures the proteins it encode are created exactly where they are need. Degrading mRNA transcripts can quickly stop the production of surplus proteins and free machinery to create what is actually needed. Protecting mRNA transcript in times of stress can enable the cell to quickly bounce back once the stress has passed. 

Unfortunately, the depth and subtlety of regulatory mechanisms acting on RNA transcripts continues to push the limits of the technology used to acquire the data used to inspect it. The volume and complexity of available data sets have surpassed the human capacity to objectively judge their validity. The dependency of computational methods enables the extraction of information on unprecedented scales, but also open the door to incorrect and unreproducible results on a scale previously unheard of. Only through the proper execution of statistical analysis and repeatable experiments can we reliably understand the complex world of RNA. This is leading to an demand for a new generation of biologist who combine specific knowledge of a biological process with broad understanding of computational and statistical methods.
\end{document}