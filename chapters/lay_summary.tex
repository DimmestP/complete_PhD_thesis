\documentclass[../main.tex]{subfiles}
\graphicspath{{\subfix{../figures/}}}

\begin{document}
\chapter{Lay Summary}
\onehalfspacing

A simple model for how cells regulate themselves begins with deoxyribonucleic acid (DNA) as the information storage molecule inherited over generations and ends with the proteins encoded by DNA that a cell uses to respond to its environment.
Ribonucleic acid (RNA) is considered to be an auxiliary molecule that is used to facilitate the flow of information from DNA to ribosomes, the cellular machinery that creates proteins from amino acids.
However, with less than 2\% of the human genome encoding a protein, the model is over-simplistic as it emphasises the regulatory role of proteins over RNA.
Even within this model, RNA is required to perform a multi-faceted role: the DNA template of a protein is transcribed as messenger RNA (mRNA) which transports it to a ribosome, the ribosomes themselves are predominately made of ribosomal RNA (rRNA), and the amino acids used to create proteins are carried by transfer RNA (tRNA).
Beyond this model populations of non-coding RNA (ncRNA) continue to be discovered with distinct regulatory roles, including: long non-coding RNAs (lncRNA), microRNAs (miRNA), and small nuclear RNAs (snRNA).

Our understanding of the world of RNA has been revolutionised over the last 40 years by technology that has enhanced the extraction and quantification of different RNA populations.
Experiments can now be designed to complete a wide range of tasks from carefully comparing specific RNA targets across large samples to exploring differences in entire populations of RNA transcripts across sub-cellular compartments. 
However, as the experiments have become more sensitive and the regulatory mechanisms of interest more subtle, the detection of biologically significant effects from experimental noise has become increasingly complex.
Therefore, the demands on molecular biologists now include: biological knowledge and experimental skills to plan and execute an experiment; and programming and statistical skills to analyse the data they create.
Biologists are equipped to meet the biological knowledge and experimental demands, but the demand for programming and statistical skills has yet to be met by sufficient training or funding.
This in turn leads biologists to depend on proprietary software or to develop their own analysis scripts without the knowledge of best practices or understanding the implicit assumptions behind the methods they use.
Closed, obscure, and incorrect analyses are fueling an ongoing crisis around reproducing results in published papers.

This thesis outlines how best practices in software development and rigorous statistical analyses can contribute to more informative and reproducible experiments investigating the regulatory role of RNA.
The thesis consists of three main results chapters.
The first results chapter describes the development of a software package called tidyqpcr which analyses data from a key experiment in molecular biology.
tidyqpcr uses comprehensive documentation and intuitive function design to empower biologists to conduct quality-controlled experiments and publish reproducible results.
The second results chapter implements tidyqpcr, together with the rigorous analysis of several other experimental assays, to detect subtle interactions between short regulatory sequences within mRNA.
The final results chapter introduces a novel statistical method to remove known biases in experiments designed to compare changes in RNA populations between sub-cellular compartments.
This research contributes to our understanding of how cells regulate themselves through their finessed control of their RNA and provides open-source software to enable other researchers to enhance their own experiments.

\end{document}