\documentclass[../main.tex]{subfiles}
\graphicspath{{\subfix{../figures/}}}

\begin{document}

\chapter{{Introduction}}

The acquisition of quality, high-throughput data sets is outpacing the creation and implementation of methods to handle and analyse it. 
In 2012, the European Bioinformatics Institute (EBI) was one of the biggest biology repositories in the world with a 20Pb storage facility \parencite{EBI2012}. 
However, by 2021, the upload of new data reached 20Pb a year with the institute having to explore collaborations with Google and Amazon in order to keep up \parencite{EBI2021}. 
The “data deluge” \parencite{Royal2012} is not only affecting infrastructure but is leading to fundamental questions about the philosophy of science as scientific publications are starting to benefit from the quality and quantity of their data rather than the significance of the conclusions they reach \parencite{Leonelli2019, Botstein2010}. 
The pursuit of big data in biology is undoubtedly leading to answers to questions thought unassailable a few years ago, such as AlphaFold base-pair resolution protein structures for entire genomes \parencite{Jumper2021}. 

Significant progress has been made in developing methods that select relevant features in the sparse high-dimensional data sets.
In investigations of linear covariates, robustness to noise and outliers can be improved by using alternative loss functions, such as the least absolute deviation, or the introduction of penalising terms, such as the $L_n$-norm of the regression coefficients \parencite{Wu2015}.
Reducing the dimensionality of data sets to emphasise regions of interest has also become standard through methods such as principle component analysis \parencite{Wall2005}.  
Methods to comparing alternative models of high-dimensional data have been explored including error estimation with cross-validation \parencite{Molinaro2005} or calculating the Akaike information criterion (AIC) \parencite{Akaike1998}.
Finally, conservative methods to test the significance of multiple features whilst maintaining power are widely implemented, e.g. the false discovery rate (FDR) \parencite{Benjamini1995}.

An alternative approach to analysing high-dimensional data sets is the application of flexible deep-learning algorithms.
A variety of ML architectures have been successfully applied to big data across biology ranging from detecting cancer to predicting gene expression \parencite{Xie2017, Liang2015, Tang2019}.
However, complex models appear to be less tolerant of experimental noise and more susceptible to confounding batch effects than simpler models \parencite{Cortes2015, Kaiser2019, Schmitt2021}.
The decisions made by deep-learning algorithms also tend to be difficult to interpret limiting their contribution to improving our understanding \parencite{Caruana2015, Ribeiro2016}.
Other probabilistic methods, such as Bayesian hierarchical models, can offer a balance between interpretability and flexibility, which will be discussed later.
However, whatever the inherent structure of the algorithm its effectiveness will be decided by the implementation by the end user. 
Unfortunately, big data scales up the issues inherent to modern biological research alongside novel solutions. 

Science is suffering from a massive reproducibility crisis. 
In 2017, Nature surveyed over 1500 scientists and found over 70\% of them tried and failed to reproduce someone else's work \parencite{Baker2016}.
The origins of the crisis come from the lack of detail in experimental protocols and misunderstanding of statistical tests.
The level at which scientists mis-used statistical tests has even led on journal to ban any reference to statistical significance \parencite{Trafimow2015}.
Fundamentally, the work by computational biologists is just as susceptible to the reproducibility crisis \parencite{Garijo2013}.
How data is preprocessed, how missing values are dealt with and which software is used can all drastically change results \parencite{Ioannidis2009}.
As the volume and types of data increases, the sparsity of biologically relevant data points increases which only decrease the likelihood of an analysis being reproducible \parencite{Chattopadhyay2019}.

There remains a gap between the computation methods used to analysis high-dimensional data and the implementation of these methods on biological data.
Those that bridge this gap are typically biologist with no formal software engineering training \parencite{Attwood2019} and little motivation to develop prototype analysis scripts into fully fledged programs \parencite{Prins2015}. 
In addition, selection of appropriate models are dependent on subject, hypothesis and quality of data \parencite{Ching2018}.
This leads to high duplication, poor reproducibility and slower overall progress \parencite{Lawlor2015}.
Only through the use of rigorous statistical methods implemented in reproducible and usable software packages can the dawn of the big data effectively combat the reproducibility crisis in biology.

Here, I discuss work using data sets from high-throughput transcriptomics experiments to expand the quantification of regulatory effects on mRNA transcripts. 
Through the development of analysis software and statistical models that incorporate both experimental noise and non-standard interactions I explore transcript localisation and the context dependence of cis-regulatory elements. 
The introduction starts by covering the basics of key transcriptomic assays that have enabled the dawn of the 'data deluge' in molecular biology. 
qPCR, microarrays and RNA-Seq are covered with emphasis on the sources of error that can be present in these experiments. 
Then, I discuss a handful of statistical methods that enable rigorous analysis of the multi-dimensional data sets with the emphasis of discovering sparse features with biological significance. 
Finally, I outline the key software development practices that have influence the work I have conducted. 
I implement these best practices in open source software development to ensure the analysis software I develop is accessible, interoperable and usable.

\section{High-Throughput RNA Assays}

\subsection{qPCR}

Polymerase chain reaction (PCR) is regarded as one of the most significant methods in molecular biology with its inventor Kary B. Mullis being co-awarded the Nobel prize in Chemistry in 1993.
The accuracy of PCR as a method to produce copies of regions of DNA in a log-linear manner has led researchers to explore its use an accurate method to quantify abundance since its invention in the 1980's \parencite{Saiki1988}.
The quantification of the rate of amplification in real time quickly followed \parencite{Holland1991}, but it wasn't until the 2000's that biochemistry and technology matured into a reliable quantitative PCR (qPCR) method \parencite{Walker2002}.
qPCR is a relatively low-throughput quantification method when compared to the other methods described here.
However, developments in microfluidics and multiplexing target probes are overcoming the bottlenecks in conducting high-throughput qPCR \parencite{Dreier2022}.

The basic principle of PCR consists of the duplication of a region on DNA that is specified by two short nucleotide sequences, called primers, that are designed to be complementary to the start and the end of the region of interest. 
A highly thermal tolerant polymerase, adapted from the bacteria \textit{Thermus aquaticus}, is then able to complete the duplication of the region by elongation of the sequence between the two primers \parencite{Saiki1988}.
The duplication cycle is repeated several times leading to the number of copies of the original region grows exponentially.
The PCR polymerase must be thermal tolerant as the duplication cycle is rapidly repeated by raising the temperature to melt the newly created complementary stand away from the original strand before dropping the temperature back down to enable the next round of elongation. 
Quantifying the rate of amplification, is done by introducing dyes that only fluoresce when a region has been successfully duplicated. 
The fluorescence of the sample is measured as the PCR cycle is repeated to determine the exponential growth in duplicates.
Quantitative PCR (qPCR) uses the amplification curve to infer the number of transcripts of the target sequence in the original sample \parencite{Holland1991}.

\subsubsection{qPCR Methods: RNA vs DNA}

qPCR is highly optimised for amplifying DNA fragments using the \textit{Thermus aquaticus} polymerase. 
Therefore, to quantify RNA fragments an additional step is required to create cDNA from RNA using a reverse transcriptase. 
Unfortunately, this step can be a significant source of variation and has been determined to be the source of most variation between RNA samples \parencite{Stahlberg2004}. 
The variation in cDNA yield between replicates can be caused by the choice of reverse transcriptase priming method, the original RNA target concentration and the total RNA concentration in the sample. 
In order for a RT-qPCR experiment to be reproducible the reverse transcriptase step must be optimised and clearly described. 

\subsubsection{qPCR Methods: Taqman vs SYBR Green}

There are two common types of fluorescent dye used to measure duplicated fragments: TaqMan and SYBR Green.
SYBR Green is a dye that fluoresces when it binds to any double stranded DNA species in the sample.
This leads to it being a cheap and relatively easy to use system, but it is highly susceptible to contamination and it is unable to distinguish between duplicates from different targets. 
TaqMan methods bind the fluorescent dye to a oligonucleotide probe that is designed to attach to the region of interest somewhere between the two primers. 
The oligonucleotide probe has the fluorescent dye on one end and a quencher on the other.
The quencher stops any excitation emissions from the dye through fluorescence resonance energy transfer. 
However, during elongation, when the polymerase reaches the oligonucleotide probe the probe is hydrolysed separating the dye from the quencher.
Emissions from the fluorescent dye can then be measured and the creation of a new duplicate is detected. 
The introduction of a custom oligonucleotide probe increases the specificity of TaqMan methods and reduces the effects of contamination. 
Also, the abundance of multiple targets can be measured in the same sample by carefully designing different oligonucleotide probes with different fluorescent dyes.
Unfortunately, the design and creation of custom probes causes TaqMan methods to be more expensive and technical. 
The accuracy of TaqMan methods is comparable to the cheaper SYBR Green methods, if correctly conducted \cite{Tajadini2014}.
Although the limit of detection (LOD) of low copy targets depends on protocol optimisation. 

\subsubsection{Quantifying Abundance: Curve Fitting vs Cycle Threshold}

The exponential limit of the number of duplicates per cycle enables methods that compare abundance across samples. 
Assuming all samples reach the exponential growth stage at the same time then the difference in fluorescence at any cycle of the PCR assay is dependent only on the original copy number. 
However, the amplification curve of duplicates per cycle is not a perfect exponential as there is a limited window through which the number of duplicates will grow exponentially.
The window is defined by limitations in detecting fluorescence at low abundance and the exhaustion of resources at high abundance.
The original copy number can be inferred from the fluorescence if you set a particular fluorescence threshold during the exponential phase and compare the number of cycles needed for a sample to reach it.
Unfortunately, this method assumes both that each sample reaches the exponential phase at the same time and that each cycle doubles the number of duplicates perfectly for each sample. 
An alternative method fits an sigmodal curve to the amplification curve and uses this model to deduce the cycles required to reach a the threshold.
The additional fitting can account for difference in the times to reach the exponential growth phase between samples and can directly account for deviations in perfect duplication \parencite{Swillens2008}. 

\subsubsection{Quantifying Abundance: Relative vs Absolute}

Multiple methods exist for converting cycle threshold measurements, Cq, into quantitative values for sample abundance whilst accounting for experimental and technical noise.
First, qPCR experiments can be designed to measure the relative change in abundance across samples.
Relative abundance measurements depend on the determination of genes that have constant expression across all samples/conditions.
Any change in the gene(s) of interest across samples can then be detected by comparing $\Delta$Cq or the expressions relative to the set of constantly expressed genes.
Normalising the fluorescence to genes with constant expression minimises batch effects introduced by sample preparation, reverse transcription and reactants.
Alternatively, the absolute number of copies of a target in a sample can be estimated.
Absolute quantification of a target requires a preliminary experiment where known initial quantities of the gene of interest are measured with qPCR.
Several amplification curves for the gene of interest, with gradually increasing copies of the gene of interest, are measured to create a collection of standard curves.
Next, the sample from the primary experiment is measured with qPCR and its amplification curve is compared to the standard curves.
The absolute copy number of the gene of interest in the experimental sample can be interpolated.

\subsection{Microarrays}

In the 1980's an assay was published to simultaneously determine multiple specific cell surface antigens through the use of matrix of antibodies fixed to a glass slide \cite{TseWen1983}.
The opportunity to quantify multiple characteristics of a sample using the same chip led other labs to explore attaching oligonucleotides to a slide, inventing microarrays \parencite{Schena1995}.
Microarrays facilitated the creation of the first high-dimensional data sets in transcriptomics. 
The technology enabled the abundance in thousands of genes to be measured simultaneously unlocking studies of gene expression regulation across conditions \parencite{Gasch2000}. 
The assay also benefited from the high quality sequencing of the genomes of multiple species as oligonucleotide probes could be designed to investigate genome-wide behaviours \cite{Lander2001}.

Microarrays consist of a glass or silicon substrate with spots of DNA printed on in a regular grid. 
Each DNA spot is a complement to a different target which fluoresces when bound with the target. 
In assays to determine differential expression, two colour microarrays are used where each spot contains two fluorescent probes; one to detect the target abundance in the sample of interest and one to detect target abundance in a control or another sample of interest.
A camera detects the level of fluorescence across each spot after excitation by a laser which is used to determine the abundance of that target.  
The two colour microarray assay measures the fluorescence of the two fluorophores and uses the ratio to determine changes in expression.
As the DNA probes have to be designed and printed onto the glass plate their complementary targets have to be decided before conducting the experiment which reduces the opportunity to discover novel regulatory elements \parencite{Schena1995}.

The creation of the high-dimensional data sets with transcript abundance of thousands of genes over dozens of conditions exposed biologists to the n $<<$ p problem.
The size of the data sets acquired were large, but the small number of data points, n, compared to the number of genes and conditions, p, led to low statistical power.
Overcoming the n $<<$ p problem has been a fruitful task for applied statistics with robust methods for sharing expression behaviour across gene and conditions \parencite{Gui2005}.
In addition, the packaging of these novel methods for shrinking variance, calculating penalised t statistics and implementing false discovery rate adjustments promoted the development of quality research software \parencite{Smyth2005, Ritchie2015}.
Distributing the analysis software for microarray arrays led to the open source bioinformatics software repository Bioconductor which has grown to include over 1000 packages for tackling analysis problems across biology \parencite{Huber2015}.


\subsection{RNA-seq}

For 30 years the primary method for sequencing DNA was Sanger sequencing with its massively successful application in decoding the human genome \cite{Lander2001}. 
The method consists of introducing a di-deoxynucleotide triphosphate (ddNTP) version of one of the four nucleic acids which induces premature termination of elongation when the polymerase incorporates it into a DNA chain.
Due to the stochasicity of elongation the introduction of ddNTP will occur at different stages of duplication leading to the creation of a population of different lengths of copies of a target.
Separating the population by weight using electrophoresis creates bands where the chosen nucleic acid has been replaced by a ddNTP version.
Repeating the process for all nucleic acids enabled a target sequence to be read off \parencite{Sanger1977}.
Similar to qPCR, Sanger sequencing can also be extended to RNA sequencing by introducing a reverse transcriptase step to make cDNA.
The accuracy of the Sanger method means it is remains in use today, but the cost and difficulty of scaling up the method has limited it use.
Instead, the success of the microarray, and its densely packed array of oligonucleotide targets fixed to a solid surface, ushered in a new sequencing technology.

\subsubsection{Shotgun Sequencing}

In the early 2000's, several companies competed to overcome the microarray's deficiency in requiring DNA targets to be defined before the experiment \parencite{Rusk2007}.
Solexa (now Illumina) developed adapters that could be ligated to any sample of DNA and facilitate the attachment of the DNA sample to a solid surface. 
Fluorescent nucleotides were created that could terminate elongation in a reversible way.
These nucleotides enabled a base-pair by base-pair cycle of elongation with the identity of the last bound nucleotide being identified by its colour.
Fixing the fragments of DNA to the surface meant islands of duplicates of the original fragment would be created and a clear fluorescent spot could be detected.
Unfortunately, this technology only allowed accurate determination of reads with length <50 nucleotides originally.
As this length is significantly shorter than the majority of sequences of biological interest, the preparation of samples for Solexa sequencing includes a step to break samples in to small fragments, coining the term shotgun sequencing.
The size of the fragments are of the order of 100 nucleotides, which is still larger than the length of reads received from the sequencing machine.
An extension to the single read per fragment method enabled paired reads, one at either end of the fragment, to be detected.
Receiving two reads per fragment improved the accuracy of determining where on the genome a fragment maps to and opened the door to exploring structural variants where fragments were copies of non-adjacent segments of the genome \parencite{Risca2015}.

\subsubsection{Long read Sequencing}

Shotgun sequencing propelled molecular biology into an age of affordable, high-throughput sequencing data.
However, the limit in read size from shotgun sequencing makes tasks such as de novo genome assembly and the detection of structural variants significantly difficult.
Long read sequencing technology has now matured with Oxford Nanopore and PacBio offering solutions to read 1000's of nucleotides at a time. 
PacBio sequencers uses fluorescent nucleotides to determine sequences similar to a Illumina sequencer, but instead of binding fragments to a solid surface they form single stranded circles from long segments of double stranded DNA (ddDNA).
Hairpin-shaped SMRTbell adapters are attached to either end of a ddDNA segment creating a closed loop than is opened up into a circle. 
A polymerase is attached to the adapters which can loop round the circle producing multiple copies of both strands of the original ddDNA segment \parencite{Hu2021}.
An Oxford Nanopore sequencer consists of a membrane with 100's of transmembrane proteins that alter their electric resistance when deformed by nucleotides moxing through them.  
A current passing through the transmembrane protein then produces a signal in response to the nucleotides moving through the protein.
Machine learning algorithm have been trained to convert the signals in the current into the sequences nucleotides \parencite{Jain2016}.

\subsubsection{Overview of RNA-Seq Based Assays}

As high-throughput RNA-Seq technology has matured, assays to explore a variety of different transcriptome effects have been developed.
poly(A) enrichment.
With around 80\% of the RNA in a cell being ribosomal, methods to investigate other RNA populations have been developed using enrichment, through poly(A) tail selection, or depletion \parencite{Stark2019}. 
Multiplexing methods now allow samples to be pooled together by introducing sample unique barcodes to read adapters which enables ultra-high-throughput methods with one library preparation stage \parencite{Craig2008}.
RNA-protein interactions can be discovered by UV cross-linking transcripts to proteins, pulling out the protein of interest, degrade the protein and analyse the remaining RNA \parencite{Granneman2009}.
Pulse labelling methods can uncover genome-wide transcript production and degradation by introducing a labelled nucleotide and measuring changes in the population of transcripts with that nucleotide \parencite{Chan2018}.
Transcript isomers created by alternative polyadenylation are uncovered by using adapters that ensure reads are anchored to the poly(A) tail \parencite{Pelechano2013}. 
Localisation of transcripts to organelles or membranes can be detected by ultra-centrifugating cell lysate and sequencing the pellet \parencite{Iserman2020}.
Finally, single-cell RNA-Seq technologies are unlocking new cell types and new sources of heterogeneity between homogeneous samples \parencite{Jovic2022}.

\subsubsection{Biases in RNA-Seq assays}

The ubiquitous use of RNA-Seq assays across biology has led to multiple advances in its accuracy and reliability, but many well documented biases remain.
The fragmentation step of RNA-Seq methods introduces a significant gene length bias as longer genes create more fragments \parencite{Oshlack2009}.
GC content of genes changes the reliability of base-calling and alters read-coverage \parencite{Dohm2008}.
RNA-Seq data sets are also highly susceptible to batch effects with total reads per run changing by orders of magnitude \parencite{Auer2010}.
The choice of RNA extraction and enrichment can lead to significant changes in differential expression detection in the same samples \parencite{Sultan2014}.
Meanwhile, poly(A) anchored reads can initiate elongation from internal stretched of adenine instead of the 3'end tail or switch templates mid-elongation \parencite{Balazs2019}.


\subsubsection{RNA-Seq Analysis}

RNA-Seq analysis consists of three core steps: Quality Control, alignment and counting. The exact quality control steps can change significantly between types of assay.
For example, enrichment of mRNA transcripts using degradation or poly(A) anchors need to be checked for effectiveness by inspecting ribosomal RNA content or tRNA levels.
Meanwhile in the case of single-cell RNA-Seq checking for cases where two or more cells may have accidentally been combined (as doublets are common in occurrence in many technique) is a vital step that is not required for bulk RNA-seq methods.
However, across all methods it will be required to check for sequence amplification biases, calling quality and whether each lane has successfully detect reads.
Once QC has been completed it is then important to remove any adapters and UMI that may have been introduced during the library preparation as these may complicate alignment to the organism genome as they will not be included. 
Also it is common to trim the 3' end of reads as more error in nucleotide callings tend to occur at then end. 
If the assay also includes multiplexed samples these need to detect and separated. Once the reads have been trimmed and multiplexed, then they need to be aligned to the organism genome in order to be able to determine which gene they correlate to.
A variety of genome aligners are available depending on organism and computing infrastructure limitations.  
BowTie2 is regularly held as the most accurate aligner but it is not splice aware.
Other aligner like star or hi-sat2 are much better at aligning reads across splicing junctions. 
There is a significant difference in speed and memory requirements between the two and users need to decide what is the priority for them.
It is vital that quality control steps are conducted after the alignment step.
Are the vast majority of reads aligned to your organism genome or could a contaminant be present. 
Visualising reads on a genome browsers is important at this point too to check you have correctly defined strandedness. 
Once the sequences have been aligned the next step is to remove PCR duplicates if UMIs are present. 
If the UMIs with random sequences are present in reads aligned to the exactly the same gene then they are considered duplicates and can be flatten to just one read. 
Finally, with the deduplicated aligned reads fully processed the user can then count the number of reads to each gene or conduct other downstream analyses such as detecting isoforms \parencite{Conesa2016}.

\subsection{RNA-Seq Analysis Pipelines}

The complexity in analysing high volume RNA-Seq data sets has lead to the development of scaleable, flexible and reproducible analysis pipelines.
Assay dependent quality control steps, from removing adapter sequences to mapping to different genome annotations, are often completed by software packages written in different scripting languages.
Workflow languages, such as Nextflow, are able to integrate the inputs and outputs of software in a domain agnostic manner \parencite{DiTommaso2017}.
A community of bioinformaticians are bringing together standardise modules using Nextflow which can be cherry picked to create the best pipeline for any specific RNA-Seq assay \parencite{Ewels2020}.
Furthermore, as differences in software versions contribute to different outcomes workflow languages are being combined with containers: such as singularity and docker \parencite{DiTommaso2015}.

\section{Post-transcription Regulation}

\subsection{RNA-binding Proteins}

\subsection{mRNA Degradation}

\subsubsection{mRNA localisation}

\section{Statistical methods}

\subsection{Linear Regression}

penalised linear regression

\subsection{Gaussian Processes}
Gaussian processes are a Bayesian method developed for analysing time series data. Time series data is structured differently to other continuous data types as it often has strong autocorrelation between neighbouring points and repeating cycles. Standard linear regression models often struggle to model this behaviour (although circular patterns can be learnt from finite linear combination of sin/cos functions). They are the projection of Gaussian distribution into infinite dimensions. By defining the covariance function, which relates the value of each point to every other point, can a various attributes to the fitted model. For example, if the fitted line must be continuous and differentiable then the covariance functions must be similarly defined. Similar to a finite multi-dimensional Gaussian it is fully defined by two parameters:  the mean function and the covariance function. As it is a Bayesian method it also create an error at in the expected value at any point. The covariance function defines how smooth the function mapping neighboring points are. You can use a number of well characterised functions as the covariance function or you can use a neural network to learn complex connections. 

The omniplate python package developed by the Swain lab uses a Gaussian process with an infinitely differentiable covariance functions. The enables the time series protein florescence data to be model, then doubly differentiated to find the point of max increase in florescence. It is important to use a similar time point to compare all the different constructs at and the time of max growth rate is appropriate. Partly because it is simple to calculate whilst making sure all the constructs are being compared at a similar point in the growth cycle (mid exponential). It also important to account for auto-fluorescence and the size of cells in various media in order to remove confounding contributions to sample fluorescence.

\subsection{Bayesian Statistical Models}

To be able to compare samples across RNA-seq runs users need to be able to remove biases that changes from run to run. For example, the amplificaition of reads from each lane of a RNA-seq machine varies significantly. Therefore direct comparison of transcript counts mapped to a genome will, for the most part, depend on the total reads read by that lane of the RNA-seq machine.  Other amplification biases can be introduced, the shotgun method of short read sequencing means the number of reads per gene will be proportional to the length of that gene. Simply because longer genes can be separated into more fragments. In addition elongation biases between the nucleotides can change the amplificiation ratio of genes. Certein gene with high GC content may not be amplified as well because they bind too tightly for the melting stage to occur efficiently. 

Raw reads are rarely used as the measure of expression between samples and between genes. Instead, they are normalised either by internal methods or by external controls such as spike-in samples. Internal normalisation commonly consists of converting raw reads into transcripts per million or reads per million. It can been widely reported at the read per million method has significant biases and should not be used. However, transcripts per million account for the total read variation between runs and the gene length biases. However, dividing by total reads does leave TPM counts vunerable to being dominated by the behaviour of a few highly expressed genes. Since there is a large order of magnitude between expression across an organisms genome, gene that are expressed on the order of $10^4$ transcripts per cell will contribute more to the total reads compared to transcript that only occur once or twice. If the experiments significant change the expression of the highly expressing genes but the users is mostly interested in the behavour of the lowly expressing genes, comparing TPM may confound expression patterns. Neither method attempts to deal with GC biases. Alternatively, an external control of known volumes of synthetic RNA can be introduced with differing GC content and lengths. The comparison of these RNA levels after being amplified in the RNA-seq assay can discover biases in transcript length, total read and GC content. A rigorous method for using this information to normal counts across the genome has not be developed as error in spike-in volumes are very common and obscure predictions expected reads post-sequencing.

\section{Research Software Engineering}

The development of high-throughput, multi-omic experiments across the biological sciences has led to an unprecedented demand on software for research.
In the late 90's less than 20\% of research papers mentioned the use of software in their research. 
In 2021, over 70\% of publications stored on pubmed cited the use of software.
Furthermore, articles accepted in the highest impact journals cited more pieces of software on average. (sentence emphasising use of statistical software). 
Software developed specifically to answer research questions has rapidly become a cornerstone of the modern empirical method \cite{Schindler2022}.
The ubiquitous use of software in modern research leads to difficulty in its definition due to its broad range of users, developers and purposes.
However, here we use the definition given by the Knowledge Exchange \parencite{KL2016}:
 
\begin{quote}
"Research software (as opposed to simply software) is software that is developed within academia and used for the purposes of research: to generate, process and analyse results.
This includes a broad range of software, from highly developed packages with significant user bases to short (tens of lines of code) programs written by researchers for their own use."
\end{quote}

Despite its importance, research software development remains a nascent subdomain of software development. 
The academic position of research software engineer was only coined in the late 2000's \cite{Prause2010} with the creation of the society of software engineers being founded in 2010. 
Exemplary research groups and institutions now exist with the primary aim of developing software to answer, or facilitate others in answering, research questions (Software Sustainability Institute, Alan Turing Institute, etc). 
However, it still remains a domain primarily occupied by self-taught programmers as an aside from their primary domain of expertise. 
The software itself is regularly developed with specific short term goals during the latter stages of scientific inquiry to provide statistical summaries and deduce significant discoveries. 
Therefore, practitioners are rarely trained in the standard development practices that are used across all other professional software development contexts.

\subsection{Software development practices}

%https://f1000research.com/articles/8-1353#ref-19

As software development has become a profession, practitioners have created practices and frameworks to facilitate efficient and rigorous software development.
The practice of don't repeat yourself (DRY) is one of the most commonly implemented to encourage programmers to write shorter and specific functions to solve regularly occurring tasks. 
Minimising the number of repetitions helps reduce errors (simply from imperfect copying), improves readability and enables faster debugging as compartmentalising tasks into separate functions enables you to test each function separately \parencite{Thomas1999}.
However, DRY does have its opponents as the focus on general abstractions can lead to unreadable code.
For example, a correctly design suite of test is intended to help diagnose bugs during the development stage.
Abstracting error messages and test to the shortest, most general form can invalid their usefulness in diagnosing bugs.
Therefore, using Descriptive and Meaningful Phrases (DAMP) and repeating how input arguments are defined within each test helps to speed the diagnosis process.

The first software development framework, and the one still most used in the sciences, is called Waterfall.
Waterfall introduces a structure to the coding practice by outlining a series of stages that are completed linearly.
It starts with a detailed specification before moving to development and finally implementation.
Alternative practices generally deviate from the waterfall ethos of leaving implementation to the very last stage.
Instead they prioritise creating a minimal viable product as soon as possible and testing its functionality.
Others even enable the specification to be altered and redefined as the project develops.
The flexibility to redefine the project as it is being developed is why the alternative practices are called agile. 

Two common agile practices in industry are scrum and extreme programming.
Scrum is the modern archetypal agile method \parencite{Schwaber2020}.
Instead of fixing the development schedule as each stage much be completed sequentially the project is broken into sprints each iteratively adding some functionality which is reviewed tested and implemented before moving onto the next.
Constantly reviewing and testing the code enables programmers to catch bugs early (ensuring complex dependency issues don't arise) receive feedback on whether initial functionality is actually useful and achievable.
As its name suggests extreme programming pushes the principles of agile programming to the extreme \parencite{Beck2004}.
Updates to the software are expected on a weekly basis with additions tested and implemented. 
In addition programmers are expected to conduct paired programming. 
Two programmers work together at all time with only one coding while the other verbally dictates what should be added. 
Although this reduces the number of lines of code written per developer, with two coders constantly reviewing each other code the number of programming bugs introduced is reduced which is assumed to negate any reduction in productivity.
Inclusion of agile practices in biomedical research suggests the iterative nature of exploratory research combines effectively with the flexibility of agile software development\parencite{kane2006}.

\subsection{Open source software development}

The growing expectation for a science that has its results open to all to read, data freely available to use and decisions that are completely transparent leads to an natural proclivity towards the already developed world of open software. 
The software development practices described above can be followed with a closed development process or an open one. 
It is fairly prolific across research to leave any code used in a research paper to be 'available upon request'. However, the encouragement of open research software practices offers many of the advantages currently seen in industry. 
The success of the Google supported Machine Learning framework TensorFlow showcases the enhanced re-usability, collaboration and accessibility enabled by adompting an Open Source framework. 

Conducting research software development in an open manner offers several benefits to the developer.
Open source projects have improved findability, accessibility, reproducibility, transparency.
Open science research overall is linked with increased citations, funding bodies placing more weight on open access policies and open project tend to get more coverage in the media \parencite{McKiernan2016}.
A modern parable for open source software development comes for the epidemiological modelling of the spread of COVID-19 by Prof Neil Ferguson at Imperial Collage London.
Crucial to the justification of national lockdowns to kerb the spread of COVID-19, the model was actual developed 13 years prior using undocumented, closed source C++ code.
After six weeks of intense revisions and refactoring, with direct help from Microsoft and GitHub software architects, is now the poster child for open source research software \url{https://github.com/mrc-ide/covid-sim}.

\subsubsection{Computation and the environment}

A fast growing aspect of software development practices is the contribution of compute to the emission of CO$_2$.
It is estimated that the IT sector will be responsible for 20\% of the world's CO$_2$ production by 2030 \parencite{CCEE2020}. 
Incorporating carbon footprint estimates into coding specifications and benchmarking published code may be the norm in the future \parencite{Lannelongue2021}.
The of use software and the versions of software may increase speed, but at the cost of increased energy requirements.
For example, HISAT2 and STAR are two commonly used sequence aligners noted for significantly different speeds and memory requirements.
Previous research has noted that memory usage is overtaking CPU as the leading energy consumer in severs.
Even just the allocation of memory to a task can significantly increase energy usage with 90\% of energy spent on maintaining memory in the background regardless of load \parencite{Karyakin2017}.
On a broader note, the standard implementations of agile software development could be questioned with weekly or daily testing considered computationally wasteful.



\subsection{GUIs or CLIs}

In the initial stages of developing any piece of software the developer decides how the user will interact with the software: point and click with a graphical user interface (GUI), or programmatically with a command line interface (CLI).
Statistical software has been developed using GUIs since the Macintosh as developers attempted to attempted to reduce the mental load of conducting statistical analyses by removing the programming requirement \parencite{ValeroMora2012}. 
The better user experience in a GUI over a CLI \parencite{Staggers2000} is paid for by larger development and maintenance costs.
A GUI also tends to become outdated as there isn't be an intuitive structure to create and expand a GUI in response to novel statistical methods and problems \parencite{Unwin2012}.
Finally, analyses using CLIs tend to be more scalable and reproducible that GUI analyses as they are difficult to intergrate with modern workflow systems. 


\subsection{Software documentation}

Software development methodologies from Waterfall to Extreme focus entirely on efficient and accurate frameworks for writing code. 
Although these methodologies differ in the timing and extent of feedback from primary stakeholders none explicitly state time for writing documentation. 
Software is only as effective as its user.
Coding practices are highly developed. Efficient coding practices, interactions between teams and automatic integration and testing of work. In practice, software is limited by the users understanding of its implementation.
Documentation practices are less developed.

% user feedback and interviews
Software developers are often unaware of their own unconscious knowledge which may create a barrier for users to engage with user. 
Also, language differences between disciplines and non-native speakers can create obstacles.

\subsection{Why is research software documentation important}

Software is intended to be a tool used to complete tasks efficiently.
It alleviates some of the mundane and repetitive tasks to enable users to focus on the complex and abstract. 
However, the intent is not always matched by the execution as users may become frustrated with design idiosyncrasies or suboptimal implementations.
In fact, researchers often dread the steps in their experimental assays that require the use of software or the analysis stages of their studies.
An opinion quickly becoming untenable as the trend trend towards data rich, high throughput studies has led even the most rudimentary analyses to depend on software.
Indeed, as statistical methodologies are still being developed to explore the vast data landscape the practice of reanalysing old experiments is only going to be more common. 
Although the demands for better computational literacy in undergrad and postgrad education is partially addressing these issues only by demanding software meet users halfway by emphasising transparency and accessibility will the problem full resolved.

In order to support reproducibility and the user experience, research software needs to be rewarded on its usability and documentation as well as its functionality.
In order assess the quality of software documentation it is useful to consider the types of documentation broadly separated by its intended audience and content. 
Previous studies have recognised three categories of documentation: documentation of decisions, why did you chose this problem to solve and why did you chose this particular method to solve it; documentation of product, what is contained within this software implementation and how do users interact with it; and documentation of technicalities, how developers created the software and how maintainers can contribute to it. 
Any software intended to be shared contains some product documentation, but few research software projects outline technical details and fewer still mention any decisions made in development \cite{Geiger2018}.

Funding bodies and journals have acknowledged across the board that research data needs to be FAIR (Findable, Accessible, Interoperable and Reusable). \textbf{evidence on uptake of fair practices} However, few guidelines have been implemented on the software packages and analysis pipelines that create/enrich/analysis these dataset to generate conclusions. Are the same FAIR condition relevant to research software? What does software that is FAIR mean? % https://www.rd-alliance.org/system/files/FAIR4RS_Principles_v0.3_RDA-RFC.pdf.

% https://www.researchsquare.com/article/rs-1239393/v1

Open source documentation as well as the software itself helps combat unconscious knowledge (knowledge that the initial developer had but forgets that not everyone knows!) Unconcious knowledge can break early stage tasks especially when the primary developer leaves and there is too much of a steep learning curve for anyone to take over.  

% https://www.software.ac.uk/resources/publications/better-software-better-research

\subsection{Why is research software documentation poor?}

The existence of multiple categories of documentation leads to a diverse set of reasons why users may find a software’s documentation insufficient.
To quantitatively detect patterns in poor software documentation \cite{Aghajani2019} scowered mailing lists, github issues and stack overflow for issues posted by users pertaining problems in documentaion. 
Common issues were based on factually incorrect statements in the documentation, sections of code/functions without any documentation at all or documentation becoming out of date with the latest package versions.
Other issues discuss the ease at which API documentation could be found and searched at all, exactly what terms meant in specific contexts and not having quality translations of documentation in other languages.
As expected, a complete lack of documentation is the most common issue but on the other extreme is dense, unintelligible documentation that is difficult to maintain and search.

Every developer knows the importance of documentation, but new software continues to be created with poor documentation.
In a 2017 GitHub survey of OSS contributors, 93\% reported that “incomplete or outdated documentation is a pervasive problem” but “60\% of contributors say they rarely or never contribute to documentation” \cite{Geiger2017}.
Software documentation typically is the least credited part of software development with little time or funds allocated to its development.
Documentation writers are first to go in times of economic difficulty. \cite{Forward2002}
However, paradoxically writing software documentation requires the most diverse set of skills and experiences to enable people from different backgrounds and knowledge to jump right in at an appropriate level \cite{Geiger2018}.
Documentation writers also envelope a larger populous than those with technical programming skills as the software developers themselves can be limited in their ability to write accessible documentation by unconscious knowledge.

The skills and interests of software developers may not naturally align with the demands of writing documentation, but perhaps the issues actually lies in the lack of community between developer, documenters and users.
Rich Bowen, Community Manager for the CentOS project,  says "There’s common wisdom in the open source world: Everybody knows that the documentation is awful, that nobody wants to write it, and that this is just the way things are. But the truth is that there are lots of people who want to write the docs. We just make it too hard for them to participate. So they write articles on Stack Overflow, on their blogs, and third-party forums. Although this can be good, it’s also a great way for worst-practice solutions to bloom and gain momentum. Embracing these people and making them part of the official documentation effort for your project has many advantages." https://labs.quansight.org/blog/2020/03/documentation-as-a-way- to-build-community/

\subsection{What are the benefits of good research software documentation?}
\begin{itemize}
    \item Documentation acts as "a resource for learning and a second role: as an advertisement for the software project" and the current health of a project.\cite{Geiger2018}
    \item Documentation supports the formation of a developer and maintainer community. Shifting knowledge acquisition away from poorly moderated forums such as biostars or stackoverflow. Extends accessibility of code by highlighting unconscious knowledge and even languages.
    \item Maybe a quick look at the top downloaded R packages on bioconductor and if that correlates with quality documentation?
    \item The quality of hardware and the underlying software dependencies of a package will also change and update, requiring regular maintenance of code for it to be continued to use. However, documentation and reporting the motivations/solutions to problems is often a contribution that last far longer.
    \item Archived and outdated software packages are still being used for the alignment of sequencing data because they are well documented. Simply saying a piece of code is old/out of date is not enough. Justification of the severe negatives of using outdated code should be highlighted in the documentation.
    \item As software get more and more specialised and refined we need to come up with a better way of accrediting work and encouraging documentation so we know exactly how the dependencies run and have a chance of fixing any future eventual bugs. (xkcd meme about software dependency and maintenance. https://xkcd.com/2347/)
\end{itemize}

\subsection{How do we improve research software documentation?}

\begin{itemize}
    \item Lessons learnt from R: vignettes, roxygen, pkgdown
    \item Documentation framework to ensure the right documentation is included and maintainable.
    \item Diataxis is a popular framework for creating documentation along its two axis of knowledge: theory vs practice and acquisition vs application. They separate software documentation into four rough types: Tutorials, practical and ap- plication knowledge; How-tos, practical and acquisition knowledge; references, theoretical and acquisition knowledge; and explaintions, theoretical and application knowledge. %https://diataxis.fr/
    \item User interviews to evaluate documentation as well as functionality
    \item Use of GitHub (issue add transparency to decisions, template issues/pull request help contributors)
    \item Including/understanding the pedagogy of software documentation
    \item Resources available on the Turing Way, open life science and Mozilla
\end{itemize}

\section{Ideas}

Bioinformatics Approaches to Gain Insights into cis-Regulatory Motifs Involved in mRNA Localization %https://link.springer.com/chapter/10.1007/978-3-030-31434-7_7

Localization elements and zip codes in the intracellular transport and localization of messenger RNAs in Saccharomyces cerevisiae
%https://wires.onlinelibrary.wiley.com/doi/10.1002/wrna.1591

% https://agupubs.onlinelibrary.wiley.com/doi/10.1002/2015EA000136 https://f1000research.c 295/v1 https://www.exascaleproject.org/event/softwaredocumentation/ https://elib.dlr.de/147388/
% https://headrush.typepad.com/creatingpassionateusers/2006/01/crashcour 
% https://www.proquest.com/docview/304387282
% https://ieeexplore.ieee.org/document/1241364 https://www.jstor.org/stable/44424301
% https://scholar.lib.vt.edu/ejournals/JOTS/Summer- Fall-2000/holmes.html



\end{document}