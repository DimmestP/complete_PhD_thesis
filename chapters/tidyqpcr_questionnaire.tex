\documentclass[../main.tex]{subfiles}
\graphicspath{{\subfix{../figures/}}}

\begin{document}
\section{tidyqpcr User Questionnaire}
\subsection{Subject questionnaire}
Estimated Time : 20 Minutes

\subsubsection{Tell me about your experience with qPCR experiments.}

Question notes:
\begin{itemize}
    \item RNA or DNA qPCR ?
    \item primers only or probe sets?
    \item How many qPCR experiments in last year/two?
    \item How many planned in next 6 months?
    \item How difficult would it be to reanalyse/repeat your own qPCR experiment?
\end{itemize}

\subsubsection{Describe how qPCR experiments are used/presented in published papers related to your research.}

Question notes:
\begin{itemize}
    \item Recount a time where you questioned results/conclusion from qPCR experments
    \item Did you attempt to reanalyse/recreate their qPCR data?
\end{itemize}

\subsubsection{Describe the design of your most recent qPCR experiment.}

Question notes:

\begin{itemize}
    \item Technical/biological/experimental replicates?
    \item Plate design?
    \item Software to design plate (excel?)
    \item Methodology for ordering samples
    \item Number of wells?
    \item Typical number of probes?
    \item Primer efficiency calculation?
    \item MIQE best practises qPCR guidelines?
    \item How did you load your plates - single-channel pipette, multichannel pipette, electronic or manual, automatic loading with what robot?
    \item What qPCR instrument did you use?
    \item How do you tell if your experiment worked - what do you do for quality control?
\end{itemize}

\subsubsection{Describe the analysis pipeline of your most recent RT-qPCR experiment.}

Question notes:

\begin{itemize}
    \item GUI / Terminal / R based?
    \item Proprietary software?
    \item See, understand and repeat every step?
    \item Customisable, paper ready output graph?
    \item Whats the biggest frustration? (is there something you know you should be doing but don’t)
    \item Would it be easy to redo an experiment (because something went wrong) using the same analysis?
    \item The features you require from qPCR software
\end{itemize}

\subsubsection{What is your previous R programming / terminal experience?}

Question notes:
\begin{itemize}
    \item Previous courses?
    \item Previous obstacles?
    \item Familiar with the concept of tidy data?
    \item Interest in learning?
\end{itemize}

\subsection{tidyqpcr worksheet}
Estimated Time : 40 minutes

\subsubsection{Follow installation instructions on \href{https://github.com/ewallace/tidyqpcr}{github.com/ewallace/tidyqpcr}.}

\subsubsection{Read through the vignette on plate designing}

\subsubsection{Create a example plate design for the following experiment:}
\begin{itemize}
    \item 8 by 12 well plate
    \item Three Biological Replicates
    \item Three Technical Replicates + “-RT” control
    \item One strain: “WT”
    \item Two conditions: + and - “menadione”
    \item Four probes: "PGK1","ALG9", " HHT2", "HTB2"
\end{itemize}

\subsubsection{Read through the instructions on conducting qPCR analysis with tidyqpcr in the multifactor vignette}

\subsubsection{Load in the example plate plan using data (tidyqpcr\_plateplan) and associated experimental data.}

\subsubsection{Normalise raw data and produce plot of differential expression under two stresses.}
\end{document}